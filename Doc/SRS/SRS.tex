\documentclass[12pt, titlepage]{article}

\usepackage{booktabs}
\usepackage{enumitem}
\usepackage{tabularx}
\usepackage{hyperref}
\hypersetup{
    colorlinks,
    citecolor=black,
    filecolor=black,
    linkcolor=red,
    urlcolor=blue
}
\usepackage[round]{natbib}

\title{SE 3XA3: Software Requirements Specification\\google-images-downloader}

\author{Team 201, CAS Dream Team
		\\ Sam Crawford, crawfs1, 400129435
		\\ Joshua Guinness, guinnesj, 400134735
		\\ Nicholas Mari, marin, 400132494
}

\begin{document}

\maketitle

\pagenumbering{roman}
\tableofcontents
\listoftables
\listoffigures

\begin{table}[bp]
\caption{\bf Revision History}
\begin{tabularx}{\textwidth}{p{3cm}p{2cm}X}
\toprule {\bf Date} & {\bf Version} & {\bf Notes}\\
\midrule
1/21/2020 & 1.0 & Added to Repo\\
1/28/2020 & 1.1 & Filled in name of project and team\\
2/08/2020 & 1.2 & Initial draft of SRS \\
\bottomrule
\end{tabularx}
\end{table}

\newpage

\pagenumbering{arabic}

This document describes the requirements for google-images-downloader.  The 
template for the Software Requirements Specification (SRS) is a subset of the Volere
template~\citep{RobertsonAndRobertson2012}.  If you make further modifications
to the template, you should explicitly state what modifications were made.

\section{Project Drivers}

\subsection{The Purpose of the Project}

The purpose of this project is to re-create an open source software application following the software development life cycle, for the software engineering course 3XA3. This project will give the group exposure and experience in how to properly develop a software product from beginning to end.

The project chosen is a google images downloader command line tool that will allow end users to download a certain number of google images given keywords. We want this product primarily to be able to help those involved in machine learning, and secondarily those involved in art.

\subsection{The Stakeholders}

The following subsections will discuss various stakeholders in the product being developed.

\subsubsection{The Client}

Dr. Asghar Bokhari, the professor of our course, and the TA responsible for marking all our deliverables, Andrew Lucentini.

\subsubsection{The Customers}

There are two main types of customers who would use the product, those involved in machine learning and those in art.

Individuals, or companies who are using machine learning algorithms or neural networks for image recognition purposes could benefit from this product as it would allow them to quickly obtain a large variety of images on a specific keyword. This can help them make their models more accurate and robust.

Artists may use this tool for projects requiring many images based off a single keyword such as collages or reference work.

\subsubsection{Other Stakeholders}

Other stakeholders include Google Images because their service is being used to scrape images off of it.

\subsection{Mandated Constraints}

\subsubsection{Schedule Constraints}

The project deliverables must be met on time according to the project description. The remaining ones are included below.

\begin{itemize}
    \item Proof of Concept Demonstration - February 11, 2020
    \item Test Plan Revision 0 - February 28, 2020
    \item Design \& Document Revision 0 - March 13, 2020
    \item Revision 0 Demonstration - March 17, 2020
    \item Final Demonstration (Revision 1) - March 31, 2020
    \item Final Documentation (Revision 1) - April 6, 2020
\end{itemize}

Further breakdown of these deliverables their respective group member assignments can be found in our Gantt Chart.

\subsubsection{Solution Constraints}

Description: The product shall be developed in Python 3.
Rationale: The existing implementation of the software is in Python 3, making it easier to re-develop if it is in the same language.
Fit criterion: The software is programmed in Python 3.

Description: The product shall get the image URLs from Google Images by scraping the HTML code returned.
Rationale: The existing implementation of the software designed its solution in this was so since we are re-developing it, that component of the solution will remain the same.
Fit criterion: The application works by scraping HTML code to get URLs

\subsubsection{Budget Constraints}

The budget for this project is \$0 so the software must be able to built without spending any money.

\subsubsection{Off-the-Shelf Software}

Since the purpose of the project is to re-implement an existing open source project, the existing version of the software will be reference heavily when developing this version. The existing software can be found at \url{https://github.com/hardikvasa/google-images-download}.

In addition to some common python libraries, urllib, urllib2, \& httplib will be used to manage and navigate URLs in the application.

\subsubsection{Partner or Collaborative Applications}

The software product being developed will use the Google Images application to download images onto the local computer. The application will use the image URL links that are embedded in the HTML code for the image returned in a Google Images search query. These URLs are stored in 'data-iurl' under the 'img' tag.

\subsubsection{Implementation Environment of the Current System}

The software product being developed will be installed on users local machines, or on servers, to function as a command line tool.

\newpage

\subsection{Naming Conventions and Terminology}

\begin{table}[ht]
\caption{\bf Naming Conventions and Terminology}
\begin{tabular}{ |c|c| } 
 \hline
 \textbf{Name/Term} & \textbf{Definition}\\ 
 The System & The product defined in this document\\
 HTML & Hypertext Markup Language \\ 
 URL & The address of a website or file on the internet \\ 
 Python 3 & The third version of the python programming language \\
 Keyword & The search term used to find images\\
 Limit & The number of images the user wished to download\\
 Google Images & The image search engine provided by Google\\
 Search Constraints & restrictions images must follow such as size or file type\\
 JPEG & A format for image files\\ 
 config file & A file that contains configurations for the system\\
 API & application programming interface\\
 GUI & Graphical User Interface\\
 \hline
\end{tabular}
\end{table}

\subsection{Relevant Facts and Assumptions}

It is assumed that users of the application have basic computer skills and are able to type commands on a command line given examples. It is also being assumed that users have basic knowledge of pictures, like what aspect ratio means, or colour type refers to.

\section{Functional Requirements}

\subsection{The Scope of the Work and the Product}

\subsubsection{The Context of the Work}

In order for the product to function as intended, the system must interact with Google Images. As a result there is a need to understand how Google Images works. This is because the original implementation of this product scrapes the HTML off of Google Images and then finds the links to the original images in the scraped HTML. Therefore, the specifics of what is required background knowledge for this product is how images are stored on the website so they can be effectively found in the scraped HTML. 

\subsubsection{Work Partitioning}

\begin{enumerate}
    \item [BE1:] The User Requests an Image.
        \begin{itemize}[wide=0pt, leftmargin=*]
            \item [Inputs:] \phantom{empty}
                \begin{itemize} [wide=0pt, leftmargin=*]
                    \item Keyword
                    \item Limit
                    \item Search Constraints
                \end{itemize}
                
            \item [Outputs:] \phantom{empty}
                \begin{itemize} [wide=0pt, leftmargin=*]
                    \item Images downloaded to computer
                \end{itemize}
            \item [Summary:] \phantom{empty}
                \begin{itemize} [wide=0pt, leftmargin=*]
                    \item Download a number of images equal to the limit of the given keyword that also match any other given search constraints.
                \end{itemize}
        \end{itemize}
\end{enumerate}

\subsubsection{Individual Product Use Cases}

The main use case of this product is the user actor requesting an image. This use case can include the user specifying optional input arguments such as a limit of images, or other search constraints such as prefixes for searching or the size of images wanted.\\

\noindent The following scenarios will outline a general interaction between a user actor and the system for both the main use case, and sub use cases. 

\begin{enumerate}[label=Scenario \arabic*:, wide=0pt, leftmargin=*]
    \item The user requests an image without specifying a keyword. The system will return a message with the proper format for an image request and remind the user that a keyword is required.
    
    \item The user requests an image with a specified keyword. The system will retrieve images from Google Images that match the specified keyword. Since a limit on images was not specified the system will download the 100 images matching the search criteria. Since an output directory was not specified, the system will download the images into a downloads directory located in the same directory where the program was executed.
    
    \item The user requests a limit of 50 images with the keyword "Cat". The system will retrieve the first 50 images off of Google Images when Cat is used as the search term. Since an output directory was not specified, the system will download the images into a downloads directory located in the same directory where the program was executed.
    
    \item The user requests a limit of 50 images with the keyword "Cat". The user specifies a download directory of "./Images/Cat/". The system will retrieve the first 50 images off of Google Images when Cat is used as the search term. Since an output directory was specified, the system will create a new directory named images then inside images create a new directory called cats. The images will then be downloaded into the new cats directory. If the images and cats directory already exists, the system will use the existing directory rather than creating a new one.
    
    \item The user requests a limit of 100 images with a specified keyword and download directory. The user also specifies specific search constraints. The system will search with the keyword and download 100 images that match the keyword and don't violate any of the search constraints. For example if the search constraint is a file type of JPEG, the system will only download JPEG images. The images will be downloaded into the specified directory, creating it if it does not already exist.
    
\end{enumerate}

\subsection{Functional Requirements}

\begin{enumerate}[label=FR \arabic*:, wide=0pt, leftmargin=*]
    \item The user shall be able to download images that match a given keyword.
    \item[Rationale:] This is the main functionality of the product. With this, users will be able to download multiple images that all match the keyword given with just one command.
    \\
    \item The user shall be able to specify the number of images they want
    \item[Rationale:] This allows the user to select how many images they want and will be a useful quality of life feature when the user knows approximately how many images they want.
    \\
    \item The user shall be able to specify a directory to download the images into
    \item[Rationale:] This allows the user to have better control of the organization of the directories on their computers and will allow the user to have better control over how they use this product.
    \\
    \item The user shall be able to specify the file type of images they desire
    \item[Rationale:] This will give the user greater freedom when using the system as they can have a set of images with a uniform file type which may help in certain situations.
    \\
    \item The user shall be able to prevent the system from downloading images from specified
    \item[Rationale:] Allowing the user to blacklist websites will give the user greater control over the system and prevent images from inappropriate sources from being downloaded in professional settings. 
    \\
    \item The user shall be able to download images only from specified websites
    \item[Rationale:] Allowing the user to white-list websites will give the user greater control over the system and prevent images from inappropriate sources from being downloaded in professional settings. 
    \\
    \item The user shall be able to prevent the system from downloading images from specified
    \item[Rationale:] Allowing the user to blacklist websites will give the user greater control over the system and prevent images from inappropriate sources from being downloaded in professional settings. 
    \\
    \item The user shall be able to add prefixes and suffixes to their image searches
    \item[Rationale:] Adding prefixes and suffixes will allow users to find the exact images they want as descriptive searches will narrow the search for the images they want. For example "Red Car" will only return red cars whereas "Car" would return images with cars of multiple different colours
    \\
    \item The user shall be able to specify the size of images they want
    \item[Rationale:] Allowing the user to specify size would be useful to artists aiming to make collages which depend on the placement and size of images used.
    \\
    \item The user shall be able to pass all search constraints, image limits and keywords through a config file.
    \item[Rationale:] Using a config file would allow groups of users to use the same settings which would be a useful features for businesses or teams using the product.

\end{enumerate}

\section{Non-functional Requirements}

\subsection{Look and Feel Requirements}

\subsection{Usability and Humanity Requirements}

\subsection{Performance Requirements}

\subsection{Operational and Environmental Requirements}

\subsection{Maintainability and Support Requirements}

\subsection{Security Requirements}

\subsection{Cultural Requirements}

\subsection{Legal Requirements}

\subsection{Health and Safety Requirements}

This section is not in the original Volere template, but health and safety are
issues that should be considered for every engineering project.

\section{Project Issues}

\subsection{Open Issues}

As the original implementation of this product relies on scraping the HTML of Google Images and finding the original image URL to download the images, there is uncertainty on if this implementation is sustainable. This is because if Google Images changes how the images are stored in the HTML, a new version must be pushed for the product to retain functionality. As Google Images has recently changed how the images are stored in the HTML, the current implementation of this project is now functional. As a result this is an issue that must be kept in mind throughout the development process.

\subsection{Off-the-Shelf Solutions}

Since the purpose of the project is to re-implement an existing open source project,  the existing version of the software will be reference heavily when developing  this  version.   The existing software can be found at https://github.com/hardikvasa/google-images-download.
\\
\\
Additionally, The implementation of this project could use an existing API or Python Library such as urllib, urllib2, httplib or SerpApi to query  and scrape Google Images.

\subsection{New Problems}

A potential problem that could arise due to this product is the unauthorized use of copyrighted materials. As the product only gets images off of Google Images, there is the possibility that a user could use a copyrighted image in a way that goes against the copyright holders wishes.

\subsection{Tasks}

The tasks for this project are the project deliverables that must be completed by their respective due dates. Detailed breakdowns of timing and due dates for the project deliverables can be found on the Gantt Chat found  \href{https://gitlab.cas.mcmaster.ca/guinnesj/google-images-downloader/blob/master/ProjectSchedule/Gantt-Chart.pdf}{here}. 
\\ \\
For a quick summary, the upcoming deliverables and their due dates are as follows:
\begin{itemize}
    \item Proof of Concept Demonstration - February 11, 2020
    \item Test Plan Revision 0 - February 28, 2020
    \item Design \& Document Revision 0 - March 13, 2020
    \item Revision 0 Demonstration - March 17, 2020
    \item Final Demonstration (Revision 1) - March 31, 2020
    \item Final Documentation (Revision 1) - April 6, 2020
\end{itemize}

\subsection{Migration to the New Product}

In order for users to migrate to the new product, they must first uninstall the previous version of the product. This is to ensure that no errors occur during the install process. Once the previous version has been uninstalled, the users can then install the new product and use it as intended.

\subsection{Risks}

A big risk for this project is the reliance on a third party platform. This is because due to the current implementation of the original product if Google Images changes how the images are stored in the HTML, the entire product looses functionality and a new version must be created and pushed before the product can be used as intended. This is a high priority risk as it has a high probability of becoming a problem during the life of the product.

\subsection{Costs}

For this project, there are two types of cost, monetary cost and cost of effort. As one of the budget constraints for this product is that we do not spend any money to produce the product, the monetary cost for this project is \$0. For the cost of effort, based on the low number of business events and functional requirements, the cost of effort is expected to be low. This could however change as requirements are refined and more work is required to make a functional product.

\subsection{User Documentation and Training}

The product will have the following documentation provided to the user to help them learn the software and guide them as they use the product.

\begin{itemize}
    \item Installation Guide to setup the product
    \item Sample Inputs with explanations of each input argument
\end{itemize}

\subsection{Waiting Room}

The creation of the GUI and functional requirements surrounding the GUI have been put on hold due to time constraints surrounding the release of this project.
\\ \\
Ideally these features will be implemented in a future release. A GUI would allow users that are less com-fordable with using the command line still have access to the product and all it's features.

\subsection{Ideas for Solutions}

As the purpose of this project is to re-implement an open source project, our solutions will be inspired by the original material. One such solution provided by the original material is scraping the HTML from google images and using the image links found through the HTML to download the images. However, due to the changes made to how images are stored on Google Images, this solution would have to be updated to accommodate for this issue.

\bibliographystyle{plainnat}

\bibliography{SRS}

\newpage

\section{Appendix}

This section has been added to the Volere template.  This is where you can place
additional information.

\subsection{Symbolic Parameters}

The definition of the requirements will likely call for SYMBOLIC\_CONSTANTS.
Their values are defined in this section for easy maintenance.


\end{document}