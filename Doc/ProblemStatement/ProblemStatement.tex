\documentclass{article}

\usepackage{tabularx}
\usepackage{booktabs}

\title{SE 3XA3: Problem Statement\\google-images-downloader}

\author{Team 201, CAS Dream Team
		\\ Sam Crawford, crawfs1, 400129435
		\\ Joshua Guinness, guinnesj, 400134735
		\\ Nicholas Mari, marin, 400132494
}

\date{}

\begin{document}

\maketitle

\section* {What problem are you trying to solve?}

Researchers, companies, artists, and hobbyists occasionally have situations where they need a large number of images related to a certain word, keyword, or topic, for example, three-legged black dogs. Although these can sometimes be obtained from a pre-existing data set, there are limitations to this approach in both the size and variety of what is out there.
\\\\One use case for this situation involves researchers or companies training machine learning algorithms or neural networks for image recognition purposes. Obtaining a large variety of images quickly is important to producing an accurate model in a decent time frame. Two other use cases include artists making a collage, or hobbyists looking to create a game, or a portfolio boosting project using machine learning.
\\\\There are currently two ways to solve the problem. The first is to manually assemble the set of images yourself. The second is to find an open dataset online that fits your needs. The problem with the first approach is that it can be extremely time-consuming and expensive, depending on the size of the data set. Although the second approach often yields great results, finding a reasonable result is less likely to be fruitful for a specific image criteria.
\\\\The original open-source program to solve this problem, google-images-download, has some limitations. Some images cannot be downloaded and are skipped, and while there are a lot of available options for filtering which images are downloaded (like colour and size), the potential exists for some more options (like specifying bit depth to work with the user’s specific image processor). Providing the user the opportunity to preview and confirm each image, if desired, will allow for easier vetting of acceptable images.


\section* {Why is this an important problem?}

This problem is important to solve because the current solutions that exist use a lot of resources, or are not very accurate. More resources mean more money and less accuracy means less quality. This makes it an issue that needs better solutions. Solving this problem by providing a quicker way to obtain a large number of images will result in products being completed quicker and more accurately. This is especially true for machine learning models where the variety of images obtained will help to ensure the model is more accurate.

\section* {What is the context of the problem you’re solving?}

The stakeholders of this problem include developers aiming to gather a large amount of sample data for training a machine learning model focused on image processing, as well as artists wishing to gather a large amount of reference material at once. To meet the requirements of all the stakeholders, a desktop environment is the best fit for the program, as it will allow the software to be used by a variety of users without the need for extra third party softwares and technologies.

\begin{table}[b]
\caption{Revision History} \label{TblRevisionHistory}
\begin{tabularx}{\textwidth}{llX}
\toprule
\textbf{Date} & \textbf{Developer(s)} & \textbf{Change}\\
\midrule
1/21/2020 & Josh & Added template file to repo\\
1/21/2020 & Sam  & Added project information\\
1/22/2020 & Josh & Added ``What problem are you trying to solve?" section\\
1/22/2020 & Sam  & Added ``Why is this an important problem?" section\\
1/22/2020 & Nick & Added ``Context of the problem" section\\
1/23/2020 & Nick & Pushed Revision 0 to GitLab\\
1/24/2020 & Josh & Reworded some phrases for increased clarity\\
1/24/2020 & Sam  & Minor improvements\\
\bottomrule
\end{tabularx}
\end{table}

\end{document}
