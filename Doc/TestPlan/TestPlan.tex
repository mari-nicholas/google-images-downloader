\documentclass[12pt, titlepage]{article}

\usepackage{booktabs}
\usepackage{enumitem}
\usepackage{lscape}
\usepackage{tabularx}
\usepackage{amssymb}
\usepackage{xcolor}
\usepackage{ulem}
\usepackage{listings}
\lstset{
  basicstyle=\ttfamily,
  columns=fullflexible,
  frame=single,
  breaklines=true,
  postbreak=\mbox{\textcolor{red}{$\hookrightarrow$}\space},
}
\usepackage{hyperref}
\hypersetup{
    colorlinks,
    citecolor=black,
    filecolor=black,
    linkcolor=red,
    urlcolor=blue
}
%\usepackage[round]{natbib}

\title{SE 3XA3: Test Plan\\Google Images Downloader}

\author{Team \#201, CAS Dream Team
		\\ Nicholas Mari, marin
		\\ Samuel Crawford, crawfs1
		\\ Joshua Guinness, guinnesj
}

\date{\today}

% \input{../Comments}

\colorlet{punct}{red!60!black}
\definecolor{background}{HTML}{EEEEEE}
\definecolor{delim}{RGB}{20,105,176}
\colorlet{numb}{magenta!60!black}

\lstdefinelanguage{json}{
    basicstyle=\normalfont\ttfamily,
    numbers=left,
    numberstyle=\scriptsize,
    stepnumber=1,
    numbersep=8pt,
    showstringspaces=false,
    breaklines=true,
    frame=lines,
    backgroundcolor=\color{background},
    literate=
     *{0}{{{\color{numb}0}}}{1}
      {1}{{{\color{numb}1}}}{1}
      {2}{{{\color{numb}2}}}{1}
      {3}{{{\color{numb}3}}}{1}
      {4}{{{\color{numb}4}}}{1}
      {5}{{{\color{numb}5}}}{1}
      {6}{{{\color{numb}6}}}{1}
      {7}{{{\color{numb}7}}}{1}
      {8}{{{\color{numb}8}}}{1}
      {9}{{{\color{numb}9}}}{1}
      {:}{{{\color{punct}{:}}}}{1}
      {,}{{{\color{punct}{,}}}}{1}
      {\{}{{{\color{delim}{\{}}}}{1}
      {\}}{{{\color{delim}{\}}}}}{1}
      {[}{{{\color{delim}{[}}}}{1}
      {]}{{{\color{delim}{]}}}}{1},
}

\begin{document}

\maketitle

\pagenumbering{roman}
\tableofcontents
\listoftables
% \listoffigures Uncomment if figures added

\begin{table}[!b]
\begin{tabularx}{\textwidth}{lllp{7.5cm}}
\toprule {\bf Date} & {\bf Name} & {\bf Version} & {\bf Notes}\\
\midrule
2/26/2020 & Nick & 1.0 & Created Document\\
2/27/2020 & Sam & 1.1 & Filled in Tests for Nonfunctional Requirements\\
2/27/2020 & Josh & 1.2 & Filled in Test Plan and started Unit Test Plan section\\
2/27/2020 & Nick & 1.2.1 & Started Tests for Functional Requirements\\
2/27/2020 & Sam & 1.3 & Filled in Purpose and Overview\\
2/28/2020 & Sam & 1.4 & Filled in Scope\\
2/28/2020 & Josh & 1.5 & Completed Tests for Proof of Concept and Unit Testing Plan sections\\
2/28/2020 & Nick & 1.6 & Completed Tests for Functional Requirements and Traceability Matrices\\
2/28/2020 & Sam & 1.6.1 & Minor consistency and formatting improvements\\
\textcolor{red}{4/05/2020} & \textcolor{red}{Joshua} & \textcolor{red}{1.7} & \textcolor{red}{Rev1 Changes: 
grammar/spelling/formatting, adding additional functional requirements, providing greater clarity for certain sections, 
got rid of first person, explicitly defined inputs/outputs for all tests.}\\
\bottomrule
\end{tabularx}
\caption{\bf Revision History}
\end{table}

\newpage

\pagenumbering{arabic}

\section{General Information}

\subsection{Purpose}
The purpose of this document is to provide an overview of the testing, validation, and verification of google-images-downloader. These test cases were designed to be used as a reference during the implementation of both the program's source code and its actual testing.

\subsection{Scope}
The scope of the testing of google-images-downloader will ensure that all testable requirements, both functional and nonfunctional, are met and that the problems found in the proof of concept are addressed. It will also confirm that each distinct part works in isolation and with its collaborators, and that the system as a whole performs its duties as expected.

\subsection{Acronyms and Definitions}
	
\begin{table}[hbp]
\label{TableOfAbbr}

\begin{tabularx}{\textwidth}{p{3cm}X}
\toprule
\textbf{Abbreviation} & \textbf{Definition} \\
\midrule
UTF-8 & Unicode Transformation Format - 8-bit\\
\bottomrule
\end{tabularx}
\caption{\textbf{Table of Acronyms}} 

\end{table}

\begin{table}[!htbp]
\label{TableOfDef}

\begin{tabularx}{\textwidth}{p{3cm}X}
\toprule
\textbf{Term} & \textbf{Definition}\\
\midrule
byte encoding & The process of encoding data into a series of bytes.\\
Gantt chart & A bar chart that illustrates a project schedule.\\
\bottomrule
\end{tabularx}
\caption{\textbf{Table of Definitions}} 

\end{table}	

\subsection{Overview of Document}

This document outlines the testing plan for google-images-downloader. This includes a description of the software, 
the team of testers, the approach to automated testing, and the tools and schedule used to optimize the testing 
efficiency and accuracy. This document also includes a description of the testing \textcolor{red}{required} for the system, 
\textcolor{red}{and} outlines the various tests for both the functional and nonfunctional requirements. This will also cover the traceability 
between the test cases and the requirements. Other tests outlined in this document are those that arose from 
the proof of concept, and those for testing each individual unit of the system.

% Mine!! - Joshua
\section{Plan} 

This section will give an overview of the testing plan, including but not limited to tools used, team, and approach.
	
\subsection{Software Description}

The software system is a google images downloader command line tool that will allow end users to download a certain number of googles images given specific keywords. The aim of this tool is to provide assistance to those involved in machine learning, and secondarily those involved in art.

The system will consist of two main subsystems. The first subsystem will be responsible for taking in user input, and constructing the google images query. The second system will be responsible for downloading the images from google and storing them. 

\subsection{Test Team}

The test team will consist of the three developers working on the project, Joshua Guinness, Sam Crawford, and Nicholas Mari.

\subsection{Automated Testing Approach}

Besides simple exploratory testing done during development for quick confirmation, the bulk of testing effort will be 
concentrated on developing three types of automated tests: unit, integration, and system. Each of these can be \sout{through} 
\textcolor{red}{thought} of as having a different level of abstraction with unit testing at the lowest level and system testing at the highest 
level. All of these will be written in pytest. \textcolor{red}{The three main aspects of the system being tested are 
simulated user input, program flow, and program logic.}

Unit tests ensure functions behave according to their design. Unit testing will also focus on boundary cases. These will be written after each module is implemented to ensure correctness.

Integration tests provide assurance that modules work together correctly and they interface properly. These will be done after a group of similar modules is complete, and after a sub-system is complete. They will also be done between sub-systems after the implementation is almost done.

Finally, system testing is concerned with ensuring that the final program adheres to the functional and non-functional requirements. This will be done after almost all of the system is complete.

Since these tests will be written in pytest, they will be run as a regression suite before a developer commits their code to GitLab. Pytest is a command line tool that will be run locally by each developer.

\subsection{Testing Tools}

One of the tools to be used is \href{https://docs.pytest.org/en/latest/}{pytest}, a unit testing framework for python. Pytest is a standalone library that will be used implementing and running the unit tests.

The other tool is \href{https://pypi.org/project/pytest-cov/}{pytest-cov} a plugin that produces coverage metrics using pytest.

The linter \href{https://flake8.pycqa.org/en/latest/}{flake8} will be used as static code quality analysis tool during development.

\subsection{Testing Schedule}
		
See \href{https://gitlab.cas.mcmaster.ca/guinnesj/google-images-downloader/-/blob/master/ProjectSchedule/Gantt-Chart.pdf}{Gantt Chart} here for the testing schedule.

Unit testing will occur parallel to development. See Section \ref{Unit Testing Plan} for further details on that. Integration testing and system testing will be done after development. All three levels of testing are visualized on the Gantt chart.

\section{System Test Description}

	
\subsection{Tests for Functional Requirements}


\subsubsection{Search Queries}

\begin{enumerate}[label=FR-SQ\arabic*:, wide=0pt, leftmargin=*]

\item \phantom{empty}

\textbf{Type:} Functional, Dynamic, Manual

\textbf{Initial State:} \sout{An empty download repository} 
\textcolor{red}{Linux machine has an empty Images folder at google-images-downloader/src/Images}

\textbf{Input:} \sout{A search request with keyword ``McMaster"} 
\textcolor{red}{The command \texttt{python3 Main.py -k "McMaster"} is provided through the command line}

\textbf{Output:} \sout{Images downloaded all related to McMaster}
\textcolor{red}{There exists a folder at google-images-downloader/src/Images/McMaster that contains
10 images related to McMaster within 15 seconds.}

\textbf{How test will be performed:} \sout{The tester will run the program to download the images and manually check if the images downloaded match the keyword given.}
\textcolor{red}{The tester will use the command line to follow the input steps above and check to ensure the output is correct.}

\item \phantom{empty}

\textbf{Type:} Functional, Dynamic, \sout{Automatic} \textcolor{red}{Automated}
					
\textbf{Initial State:} \sout{An empty download repository}
\textcolor{red}{Linux machine has an empty Images folder at google-images-downloader/src/Images}
					
\textbf{Input:} \sout{A search request with only a specific file type requested}
\textcolor{red}{The below json object which specifies the gif file type}
\begin{lstlisting}[language=json,firstnumber=1]
{
	"keyword": "default",
	"limit": 3,
	"safesearch": True,
	"directory": path.join(curdir, "Images"),
	"filetype": "gif",
	"colour": "",
	"license": "",
	"imagetype": "",
	"imageage": "",
	"aspectratio": "",
	"imagesize": "",
	"serverhost": "",
	"serverusername": "",
	"serverpassword": "",
	"whitelist": "",
	"blacklist": "",
	"region": ""
}
\end{lstlisting}
					
\textbf{Output:} \sout{A download repository with only files of the specified type downloaded}
\textcolor{red}{A folder exists at google-images-downloader/src/Images/default that contains
three images with the file extension .gif}
					
\textbf{How test will be performed:} An automated test case will send the request to the system and check that the file extensions on each of the files downloaded are correct.

\item \phantom{empty}

\textbf{Type:} \textcolor{red}{Functional, Dynamic, Automated}
					
\textbf{Initial State:} \textcolor{red}{Linux machine has an empty Images folder at google-images-downloader/src/Images}
					
\textbf{Input:} \textcolor{red}{The below json object which specifies the jpg file type}
\begin{lstlisting}[language=json,firstnumber=1]
{
	"keyword": "default",
	"limit": 3,
	"safesearch": True,
	"directory": path.join(curdir, "Images"),
	"filetype": "jpg",
	"colour": "",
	"license": "",
	"imagetype": "",
	"imageage": "",
	"aspectratio": "",
	"imagesize": "",
	"serverhost": "",
	"serverusername": "",
	"serverpassword": "",
	"whitelist": "",
	"blacklist": "",
	"region": ""
}
\end{lstlisting}
					
\textbf{Output:} \textcolor{red}{A folder exists at google-images-downloader/src/Images/default that contains
three images with the file extension .jpg}
					
\textbf{How test will be performed:} \textcolor{red}{An automated test case will send the request 
to the system and check that the file extensions on each of the files downloaded are correct.}

\item \phantom{empty}

\textbf{Type:} \textcolor{red}{Functional, Dynamic, Automated}
					
\textbf{Initial State:} \textcolor{red}{Linux machine has an empty Images folder at google-images-downloader/src/Images}
					
\textbf{Input:} \textcolor{red}{The below json object which specifies the png file type}
\begin{lstlisting}[language=json,firstnumber=1]
{
	"keyword": "default",
	"limit": 3,
	"safesearch": True,
	"directory": path.join(curdir, "Images"),
	"filetype": "png",
	"colour": "",
	"license": "",
	"imagetype": "",
	"imageage": "",
	"aspectratio": "",
	"imagesize": "",
	"serverhost": "",
	"serverusername": "",
	"serverpassword": "",
	"whitelist": "",
	"blacklist": "",
	"region": ""
}
\end{lstlisting}
					
\textbf{Output:} \textcolor{red}{A folder exists at google-images-downloader/src/Images/default that contains
three images with the file extension .png}
					
\textbf{How test will be performed:} \textcolor{red}{An automated test case will send the request 
to the system and check that the file extensions on each of the files downloaded are correct.}

\item \phantom{empty}

\textbf{Type:} Functional, Dynamic, \sout{Manual} \textcolor{red}{Automated}
					
\textbf{Initial State:} \sout{An empty download repository}
\textcolor{red}{Linux machine has an empty Images folder at google-images-downloader/src/Images}
					
\textbf{Input:} \sout{A search query with a request to not download from a specified website}
\textcolor{red}{The json object below which specifies the blacklist as "mcmaster.ca"}
\begin{lstlisting}[language=json,firstnumber=1]
{
	"keyword": "mcmaster engineering",
	"limit": 3,
	"safesearch": True,
	"directory": path.join(curdir, "Images"),
	"filetype": "",
	"colour": "",
	"license": "",
	"imagetype": "",
	"imageage": "",
	"aspectratio": "",
	"imagesize": "",
	"serverhost": "",
	"serverusername": "",
	"serverpassword": "",
	"whitelist": "",
	"blacklist": "mcmaster.ca",
	"region": ""
}
\end{lstlisting}
					
\textbf{Output:} \sout{A download repository full with images, with no images from the blacklisted website}
\textcolor{red}{A list of URLS, none of which contain ``mcmaster.ca" in the URL}
					
\textbf{How test will be performed:} \sout{The tester will manually start the search query with all 
the specified inputs and check each downloaded image URL to ensure that it does not come from 
the blacklisted website.}
\textcolor{red}{An automated test case will send the request to the system and check that the list 
of URLs to be downloaded do not contain the blacklisted site.}

\item \phantom{empty}

\textbf{Type:} Functional, Dynamic, \sout{Manual} \textcolor{red}{Automated}
					
\textbf{Initial State:} \sout{An empty download repository}
\textcolor{red}{Linux machine has an empty Images folder at google-images-downloader/src/Images}
					
\textbf{Input:} \sout{A search query with a request to only download from a specified website}
\textcolor{red}{The json object below which specifies the whitelist as ``mcmaster.ca"}
\begin{lstlisting}[language=json,firstnumber=1]
{
	"keyword": "engineering",
	"limit": 3,
	"safesearch": True,
	"directory": path.join(curdir, "Images"),
	"filetype": "",
	"colour": "",
	"license": "",
	"imagetype": "",
	"imageage": "",
	"aspectratio": "",
	"imagesize": "",
	"serverhost": "",
	"serverusername": "",
	"serverpassword": "",
	"whitelist": "mcmaster.ca",
	"blacklist": "",
	"region": ""
}
\end{lstlisting}

\textbf{Output:} \sout{A download repository full with images, with no images from the whitelisted website}
\textcolor{red}{A list of URLS, all of which contain ``mcmaster.ca" in the URL}

\textbf{How test will be performed:} \sout{The tester will manually start the search query with all the specified 
inputs and check each downloaded image URL to ensure that it comes from the white-listed website.}
\textcolor{red}{An automated test case will send the request to the system and check that the list 
of URLs to be downloaded only contain the whitelisted site.}

\item \phantom{empty}

\textbf{Type:} \sout{Functional, Dynamic, Manual}
					
\textbf{Initial State:} \sout{An empty download repository}
					
\textbf{Input:} \sout{A search request with prefixes and suffixes given}
					
\textbf{Output:} \sout{A download repository filled with images that match the search query and prefix or suffix}
					
\textbf{How test will be performed:} \sout{The tester will manually ensure that each image matches both the search query and the prefixes and suffixes given.}

\item \phantom{empty}

\textbf{Type:} Functional, Dynamic, \sout{Automatic} \textcolor{red}{Automated}
					
\textbf{Initial State:} \sout{An empty download repository}
\textcolor{red}{Linux machine has an empty Images folder at google-images-downloader/src/Images}
					
\textbf{Input:} \sout{A search request with the size of desired images specified} 
\textcolor{red}{The below json object which specifies a specific image size}
\begin{lstlisting}[language=json,firstnumber=1]
{
    "keyword": "homestuck",
    "limit": 5,
    "safesearch": True,
    "directory": path.join(curdir, "Images"),
    "filetype": "jpg",
    "colour": "",
    "license": "",
    "imagetype": "",
    "imageage": "",
    "aspectratio": "",
    "imagesize": ">2MP",
    "serverhost": "",
    "serverusername": "",
    "serverpassword": "",
    "whitelist": "",
    "blacklist": "",
    "region": ""
}
\end{lstlisting}
					
\textbf{Output:} \sout{A download repository filled with images of the specified size}
\textcolor{red}{A folder exists at google-images-downloader/src/Images/homestuck that contains
images of the specified size}
					
\textbf{How test will be performed:} \sout{An automated test case will check the size of each image to ensure it 
matches the size specified in the search request.}
\textcolor{red}{An automated test will send a request to the system, and after all images are downloaded, 
google-images-downloader/src/Images/homestuck will be iterated through to check that each file matches 
the specified size.}

\item \phantom{empty}

\textbf{Type:} Functional, Dynamic, \sout{Manual} \textcolor{red}{Automated}
					
\textbf{Initial State:} \sout{An empty download repository}
\textcolor{red}{Linux machine has an empty Images folder at google-images-downloader/src/Images}
					
\textbf{Input:} \sout{A search request with all constraints and keywords in a configuration file provided to 
the system when the user makes the request}
\textcolor{red}{The below json object and a second json object which is stored at 
google-images-downloader/src/Test/ConfigForTesting/homestuckFile.txt}
\begin{lstlisting}[language=json,firstnumber=1]
{
    "keyword": "homestuck",
    "limit": 5,
    "safesearch": True,
    "directory": "",
    "filetype": "gif",
    "colour": "green",
    "license": "",
    "imagetype": "",
    "imageage": "",
    "aspectratio": "",
    "imagesize": "",
    "serverhost": "",
    "serverusername": "",
    "serverpassword": "",
    "whitelist": "",
    "blacklist": "",
    "region": "Canada"
}
\end{lstlisting}
					
\textbf{Output:} \sout{Download repository filled with images}
\textcolor{red}{Two lists of URLs, one built from the above json object, and the one built from the homestuckFile.txt file}
					
\textbf{How test will be performed:} \sout{The tester will manually ensure that all the images downloaded match 
the keyword and search constraints given in the configuration file.}
\textcolor{red}{Automated test which will build two lists of URLs, one from the json object above, and the other 
from the textfile, then will compare the two lists to ensure that they are equal.}

\item \phantom{empty}

\textbf{Type:} \textcolor{red}{Functional, Dynamic, Manual}

\textbf{Initial State:} \textcolor{red}{Linux machine has an empty Images folder at google-images-downloader/src/Images}

\textbf{Input:} \textcolor{red}{The command \texttt{python3 Main.py -k "fruit" -l 3 -c "red"} is provided through the command line}

\textbf{Output:} \textcolor{red}{There exists a folder at google-images-downloader/src/Images/fruit that contains 3 images 
which are red in color}

\textbf{How test will be performed:} \textcolor{red}{The tester will use the command line to follow the input steps above and 
check to ensure the output is correct.}

\item \phantom{empty}

\textbf{Type:} \textcolor{red}{Functional, Dynamic, Manual}

\textbf{Initial State:} \textcolor{red}{Linux machine has an empty Images folder at google-images-downloader/src/Images}

\textbf{Input:} \textcolor{red}{The command \texttt{python3 Main.py -k "waterfall" -l 3 -ar "panoramic"} is provided through the command line}

\textbf{Output:} \textcolor{red}{There exists a folder at google-images-downloader/src/Images/waterfall that contains 3 images 
which which are panoramas}

\textbf{How test will be performed:} \textcolor{red}{The tester will use the command line to follow the input steps above and 
check to ensure the output is correct.}

\end{enumerate}


\subsubsection{Download Images}

\begin{enumerate}[label=FR-DL\arabic*:, wide=0pt, leftmargin=*]

\item \phantom{empty}

\textbf{Type:} Functional, Dynamic, \sout{Automatic} \textcolor{red}{Automated}

\textbf{Initial State:} \sout{An empty download repository}
\textcolor{red}{Linux machine has an empty Images folder at google-images-downloader/src/Images}

\textbf{Input:} \sout{A search request with a limit of 50 images specified}
\textcolor{red}{The below json object which specifies a limit of 40 images}
\begin{lstlisting}[language=json,firstnumber=1]
{
    "keyword": "software",
    "limit": 40,
    "safesearch": True,
    "directory": path.join(curdir, "Images"),
    "filetype": "",
    "colour": "",
    "license": "",
    "imagetype": "",
    "imageage": "",
    "aspectratio": "",
    "imagesize": "",
    "serverhost": "",
    "serverusername": "",
    "serverpassword": "",
    "whitelist": "",
    "blacklist": "",
    "region": ""
}
\end{lstlisting}

\textbf{Output:} \sout{A download repository contains 50 images in it}
\textcolor{red}{A folder exists at google-images-downloader/src/Images/software that contains 40 images}

\textbf{How test will be performed:} \sout{An automatic test case will run the program with the specified input 
and count the number of new files in the download repository to ensure it is correct.}
\textcolor{red}{An automated test will send a request to the system, and after all images are downloaded, 
google-images-downloader/src/Images/software will be iterated through to check that there are 40 images.}

\item \phantom{empty}

\textbf{Type:} Functional, Dynamic, \sout{Automatic} \textcolor{red}{Automated}
					
\textbf{Initial State:} \sout{An empty download repository}
\textcolor{red}{Linux machine has an empty Images folder at google-images-downloader/src/Images}
					
\textbf{Input:} \sout{A download request with a specified download folder}
\textcolor{red}{The below json object which specifies a download folder}
\begin{lstlisting}[language=json,firstnumber=1]
{
    "keyword": "software",
    "limit": 2,
    "safesearch": True,
    "directory": path.join(curdir, "newDirec"),
    "filetype": "",
    "colour": "",
    "license": "",
    "imagetype": "",
    "imageage": "",
    "aspectratio": "",
    "imagesize": "",
    "serverhost": "",
    "serverusername": "",
    "serverpassword": "",
    "whitelist": "",
    "blacklist": "",
    "region": ""
}
\end{lstlisting}
					
\textbf{Output:} \sout{Images downloaded to the specified folder}
\textcolor{red}{A folder exists at google-images-downloader/src/newDirec/software that contains 2 images}
					
\textbf{How test will be performed:} \sout{An automated test case will request a search with the 
specified parameters and automatically check to ensure that the images were downloaded to 
the folder the test case specified in the search.}
\textcolor{red}{An automated test will send a request to the system, and after all images are downloaded, 
google-images-downloader/src/newDirec/software will be iterated through to check that there are 2 images.}

\item \phantom{empty}

\textbf{Type:} \textcolor{red}{Functional, Dynamic, Manual}

\textbf{Initial State:} \textcolor{red}{Linux machine has an empty Images folder at google-images-downloader/src/Images}

\textbf{Input:} \textcolor{red}{The command \texttt{python3 Main.py -k "McMaster" -l 3 -s "moore.mcmaster.ca" -u "guinnesj -p "password"}
is provided through the command line}

\textbf{Output:} \textcolor{red}{There exists a folder at moore.mcmaster.ca:/u50/guinnesj/McMaster that contains 3 images within 15 seconds}

\textbf{How test will be performed:} \textcolor{red}{The tester will use the command line to follow the input steps above and 
check to ensure the output is correct by logging into the server.}


\end{enumerate}


\subsection{Tests for Nonfunctional Requirements}

\subsubsection{Look and Feel Requirements}
\label{LaFReqs}
		
\paragraph{Appearance Requirements}

\begin{enumerate}[label=NFR-AR\arabic*:, wide=0pt, leftmargin=*]

\item \phantom{empty}

\textbf{Type:} Functional, Dynamic, Manual, Acceptance
					
\textbf{Initial State:} \sout{A clean version of the program} \textcolor{red}{Program is not running.}
					
\textbf{Input/Condition:} \sout{A typical usage of the system} 
\textcolor{red}{\textcolor{red}{The command \texttt{python3 Main.py -k "waterfall" -l 3} is provided through the command line}}
					
\textbf{Output/Result:} Output messages displayed on the console \textcolor{red}{are fully visible}
					
\textbf{How test will be performed:} \sout{The tester will use a survey to verify that at least 80\% of users 
find that the output messages fit the console nicely.}
\textcolor{red}{The tester will use the command line to follow the input step above and check to ensure the output is correct.}

\end{enumerate}

\subsubsection{Usability and Humanity Requirements}
\label{UaHReqs}
		
\paragraph{Ease of Use Requirements}

\begin{enumerate}[label=NFR-EUR\arabic*:, wide=0pt, leftmargin=*]

\item \phantom{empty}

\textbf{Type:} Functional, Dynamic, Manual, Acceptance
					
\textbf{Initial State:} \sout{A clean version of the program} \textcolor{red}{Program is not running.}
					
\textbf{Input/Condition:} \sout{A typical usage of the system} 
\textcolor{red}{\textcolor{red}{The command \texttt{python3 Main.py -k "waterfall" -l 3} is provided through the command line}}
					
\textbf{Output/Result:} \sout{A folder of the desired downloaded images}
\textcolor{red}{A folder exists at google-images-downloader/src/Images/waterfall which has three images}
					
\textbf{How test will be performed:} The tester will use a survey to verify that at least 90\% of users are able to successfully download their images after reading the sample inputs and explanations for the first time.

\end{enumerate}

\paragraph{Learning Requirements}

\begin{enumerate}[label=NFR-LR\textcolor{red}{2}:, wide=0pt, leftmargin=*]

\item \phantom{empty}

\textbf{Type:} Structural, Static, Manual, Acceptance
					
\textbf{Initial State:} A computer without the system installed
					
\textbf{Input/Condition:} \sout{A typical installation of the system from the command line}
\color{red}A series of commands entered through the command line:
\begin{lstlisting}[language=bash]
  $ git clone https://gitlab.cas.mcmaster.ca/guinnesj/google-images-downloader
  $ cd google-images-downloader/src
  $ make install
\end{lstlisting}
\color{black}
					
\textbf{Output/Result:} \sout{A folder of the desired downloaded images}
\textcolor{red}{Program exists on local system and dependencies are installed.}
					
\textbf{How test will be performed:} The tester will use a survey to verify that at least 90\% of users that are comfortable using the command line are able to download the program within five minutes.

\end{enumerate}

\subsubsection{Performance Requirements}
\label{PerfReqs}
		
\paragraph{Robustness Requirements}

\begin{enumerate}[label=NFR-RR\arabic*:, wide=0pt, leftmargin=*]

\item \phantom{empty}

\textbf{Type:} Functional, Dynamic, Manual, Acceptance
					
\textbf{Initial State:} \sout{A clean version of the program} \textcolor{red}{Program is not running.}
					
\textbf{Input/Condition:} \sout{An invalid input; an input that would result in an error}
\textcolor{red}{The command \texttt{python3 Main.py} is provided to the command line.}
					
\textbf{Output/Result:} A descriptive error message showing the possible command line arguments is shown
					
\textbf{How test will be performed:} \sout{The tester will use a survey to verify that at least 75\% of users can identify their mistake from the error message.}
\textcolor{red}{The tester will run the input steps above and confirm the output is true.}

\end{enumerate}

\subsubsection{Operational and Environmental Requirements}
		
\paragraph{Requirements for Interacting with Adjacent Systems}

\begin{enumerate}[label=NFR-IAR\arabic*:, wide=0pt, leftmargin=*]

\item \phantom{empty}

\textbf{Type:} Functional, Dynamic, Manual
					
\textbf{Initial State:} \sout{A clean version of the program} \textcolor{red}{Most recently updated system}
					
\textbf{Input/Condition:} \sout{A typical usage of the program}
\textcolor{red}{The command \texttt{python3 Main.py -k mcmaster} is provided to the command line}
					
\textbf{Output/Result:} \sout{A folder of the desired downloaded images}
\textcolor{red}{A folder exists at google-images-downloader/src/images with 10 images.}
					
\textbf{How test will be performed:} \sout{The tester will manually verify that the program can download 
all the images using the latest format.}
\textcolor{red}{The tester will run the input steps above and confirm the output is true.}

\end{enumerate}

\subsubsection{Maintainability and Support Requirements}
		
\paragraph{Adaptability Requirements}

\begin{enumerate}[label=NFR-ADR\arabic*:, wide=0pt, leftmargin=*]

\item \phantom{empty}

\textbf{Type:} Functional, Dynamic, Manual
					
\textbf{Initial State:} The most up to date version of the program on a Linux, Windows, and Mac platform.
					
\textbf{Input/Condition:} \sout{A typical usage of the program}
\textcolor{red}{The command \texttt{python3 Main.py -k mcmaster} is provided to the command line}
					
\textbf{Output/Result:} \sout{A folder of the desired downloaded images}
\textcolor{red}{A folder exists at google-images-downloader/src/images with 10 images.}
					
\textbf{How test will be performed:} The tester will manually verify that \sout{all test cases pass on all platforms} 
\textcolor{red}{given the input, the output is true on all platforms}

\end{enumerate}

\subsubsection{Legal Requirements}
		
\paragraph{Standards Requirements}

\begin{enumerate}[label=NFR-SR\arabic*:, wide=0pt, leftmargin=*]

\item \phantom{empty}

\textbf{Type:} Structural, Static, Manual
					
\textbf{Initial State:} \sout{A clean version of the program}
\textcolor{red}{The most up to date version of the program on a Linux, Windows, and Mac platform.}
					
\textbf{Input/Condition:} \sout{An execution of the linter}
\textcolor{red}{The command \texttt{make lint} is provided to the command line under the src/ directory}
					
\textbf{Output/Result:} A successful linting with no errors to report
					
\textbf{How test will be performed:} The tester will perform a code review to manually verify that the code matches Google's coding style and that the linter returns no errors.

\end{enumerate}
\newpage

\subsection{Traceability Between Test Cases and Requirements}
\begin{table}[!ht]
\begin{center}
\begin{tabular}{|l|c|c|c|c|c|c|c|c|c|}
\hline
\multicolumn{1}{|c|}{\textbf{Test Id}} & \multicolumn{9}{c|}{\textbf{Requirement}}   \\ \hline
\textbf{} & FR1 & FR2 & FR3 & FR4 & FR5 & FR6 & FR7 & FR8  & FR9 \\
\hline
FR-SQ1 & \checkmark & \multicolumn{1}{r|}{} & & & & & & &\\ 
\hline
FR-SQ2 &  &  &  & \checkmark &  &  &  &  &  \\ \hline
FR-SQ3        &  &  &  &  & \checkmark &  &  &  &  \\ \hline
FR-SQ4        &  &  &  &  &  & \checkmark &  &  &  \\ \hline
FR-SQ5        &  &  &  &  &  &  & \checkmark &  &  \\ \hline
FR-SQ6        &  &  &  &  &  &  &  & \checkmark &  \\ \hline
FR-SQ7        &  &  &  &  &  &  &  &  & \checkmark \\ \hline
FR-DL1        &  & \checkmark &  &  &  &  &  &  &  \\ \hline
FR-DL2        &  &  & \checkmark &  &  &  &  &  &  \\ \hline
\end{tabular}
\end{center}
\caption{\textbf{Functional Requirements Traceabillity Matrix}}
\end{table}

\begin{table}[!ht]
\begin{center}
\begin{tabular}{|l|c|c|c|c|c|c|c|}
\hline
\multicolumn{1}{|c|}{\textbf{Test Id}} & \multicolumn{7}{c|}{\textbf{Requirement}}   \\ \hline
         & AR1  & EUR1 & LR1  & RR1  & IAR1 & ADR1 & SR1 \\ \hline
NFR-AR1  & \checkmark &  &  &  &  &  &  \\ \hline
NFR-EUR1 &  & \checkmark &  &  &  &  &  \\ \hline
NFR-LR1  &  &  & \checkmark &  &  &  &  \\ \hline
NFR-RR1  &  &  &  & \checkmark &  &  &  \\ \hline
NFR-IAR1 &  &  &  &  & \checkmark &  &  \\ \hline
NFR-ADR1 &  &  &  &  &  & \checkmark &  \\ \hline
NFR-SR1  &  &  &  &  &  &  & \checkmark \\ \hline
\end{tabular}
\end{center}
\caption{\textbf{Nonfunctional Requirements Traceabillity Matrix}}
\end{table}


The following Nonfunctional Requirements are not included in the Traceabillity Matrix as they do not have a test case associated with them:
\begin{itemize}
    \item \sout{LR2}
    \item PDR1: \textcolor{red}{Simply a true or false question of whether the git repository is public.}
    \item MSR1: \textcolor{red}{Concerns a future development strategy therefore cannot be tested.}
    \item MSR2: \textcolor{red}{Concerns a future development strategy therefore cannot be tested.}
    \item ACR1: \textcolor{red}{Simply a true or false question of whether the git repository is public.}
    \item CR1:  \textcolor{red}{Impossible to confirm with certainty but best efforts were made by developers.}
    \item CPR1: \textcolor{red}{Simply a true or false question of whether the README.md contains the statement.}
    \item \textcolor{red}{HSR1:} \textcolor{red}{Protection of our eyes.} 
\end{itemize}

% Mine - Joshua
\section{Tests for Proof of Concept}

There were two main issues that \sout{we ran into} \textcolor{red}{occurred} while implementing the proof of concept 
that required significant time to solve.

The first is that the Google Images HTML code structure changed from when the original project was first confirmed 
to run properly, and the start of implementing the proof of concept. This meant that the section of the original 
project regarding how the HTML is scra\sout{p}ped and the images downloaded was almost useless. To solve the issue, 
\sout{we first figured out what} the new HTML structure was \textcolor{red}{first figured out} by using 
``Inspect Element" on \sout{images.} \textcolor{red}{the page of google image results.} This also took some time as 
there are multiple links within each image subsection. Once \sout{we} it was figured out where the image was 
stored in the HTML, \sout{we could then scrape} it could then be scraped and parse\textcolor{red}{d} \sout{it} to download it.

The second issue occurred after downloading the raw HTML code and trying to parse it for the image links. 
It would not parse properly, even after it was confirmed that the HTML downloaded contained the links or phrases 
that were being searched for. \sout{We solved the issue} \textcolor{red}{The issue was solved} by going through 
each step of the code slowly and printing things out to the console. \sout{We} \textcolor{red}{It was} discovered that 
the raw HTML downloaded was actually byte encoded, as evidence by the ``b" in front of the string. 
Adding a statement to decode it to UTF-8 solved the issue.
	
\section{Unit Testing Plan}
\label{Unit Testing Plan}

This section will go over the unit testing plan for this project including the plan for the testing of internal functions, and output files. The tool pytest will be used for all unit testing.
		
\subsection{Unit testing of internal functions}

The purpose of unit testing is to ensure that a specific unit of software performs as intended. On a software function level, this is done by sending an input to a function, and asserting that the output is the expected output of the function.

For this project, after every module is complete, it will be unit tested before the developer starts work on the next module. This means that unit testing will be done during the development process, not all at the end. The developers can then work from a place of confidence and assurance that previous code written functions as intended.

\textcolor{red}{The types of unit tests implemented by the developer will be grey box tests. Grey box tests are a
combination of white box and black box where the goal is to exploit knowledge of the system to test interfaces and 
internal functionality}

The coverage metrics we will be using are \sout{85\%} \textcolor{red}{80\%} statement coverage and 
80\% branch coverage. This data will be found using the pytest-cov tool. There are no stubs 
or drivers needed for testing.
		
\subsection{Unit testing of output files}

The only output files produced are the downloaded images themselves. The contents of the file will not be unit 
tested at all; unit testing will revolve around the properties (eg. \sout{name, aspect ratio, bit-depth} \textcolor{red}{
file type, size}) of the images and the number of images downloaded.

%\bibliographystyle{plainnat}

%\bibliography{SRS}

\newpage

\section{Appendix}

\subsection{Symbolic Parameters}

The test cases listed here do not call for the use of any symbolic constants.

\subsection{Usability Survey Questions}

In addition to relevant questions for testing the appropriate nonfunctional requirements (see Sections \ref{LaFReqs}, \ref{UaHReqs}, and \ref{PerfReqs}), users will also be asked to confirm that no part of the product (eg. naming conventions) are offensive to them.

\end{document}