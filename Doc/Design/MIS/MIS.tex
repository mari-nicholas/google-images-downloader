\documentclass{article}
\usepackage{fullpage}
\usepackage[round]{natbib}
\usepackage{multirow}
\usepackage{booktabs}
\usepackage{tabularx}
\usepackage{graphicx}
\usepackage{amsfonts}
\usepackage{amsmath}
\usepackage{float}
\usepackage{hyperref}
\hypersetup{
    colorlinks,
    citecolor=black,
    filecolor=black,
    linkcolor=red,
    urlcolor=blue
}
\usepackage{soul,xcolor}

\title{SE 3XA3: Module Interface Specification\\Google Images Downloader}

\author{Team \#201, CAS Dream Team
		\\ Sam Crawford, crawfs1, 400129435
		\\ Joshua Guinness, guinnesj, 400134735
		\\ Nicholas Mari, marin, 400132494
}

%\date{\today}

\begin{document}

\maketitle

% \pagenumbering{roman}
% \tableofcontents
% \listoftables
% \listoffigures

% \newpage

\pagenumbering{arabic}

\section*{Introduction}

This document shows the complete specification for implementing all modules of the Google Images Downloader program.

\begin{table}[bp]
\begin{tabularx}{\textwidth}{llll}
\toprule {\bf Date} & {\bf Name} & {\bf Version} & {\bf Notes}\\
\midrule
1/21/2020 & Joshua & 1.0 & Created document\\
3/11/2020 & Sam & 1.1 & Set up format of document\\
3/11/2020 & Sam & 1.1.1 & Set up skeleton of each module\\
3/11/2020 & Joshua & 1.1.2 & Input and Navigate Page Module\\
3/12/2020 & Nick & 1.1.3 & Search Query and Main Module\\
3/13/2020 & Sam & 1.1.4 & Output Module\\
3/13/2020 & Joshua, Sam, Nick & 1.2 & General Edits\\
\color{red}4/6/2020 & \color{red}Sam & 
\color{red}2.0 & \color{red}Updated for Rev 1\\
\bottomrule
\end{tabularx}
\caption{\bf Revision History}
\end{table}

\newpage

\section*{Input Format Module}

\subsection* {Module}

Input

\subsection* {Uses}

None

\subsection* {Syntax}

\subsubsection* {Exported Types}

None

\subsubsection* {Exported Constants}

None

\subsubsection* {Exported Access Programs}

\begin{tabular}{| l | l | l | p{5cm} |}
\hline
\textbf{Routine name} & \textbf{In} & \textbf{Out} & \textbf{Exceptions}\\
\hline
userInput & \st{N/A}\color{red}user input from keyboard & dict & InvalidParam, MissingParam \\
\hline

\end{tabular}

\subsection* {Semantics}

\subsubsection* {Environmental Variables}

\begin{itemize}
\item Keyboard \color{red}- the user will enter flags, usually followed by values, that will be parsed to provide the input arguments to the program. Valid parameters are given in Table \ref{tab:inputParams}.\color{black}
\end{itemize}

\subsubsection* {State Variables}

None

\subsubsection* {State Invariant}

None

\subsubsection* {Assumptions \& Design Decisions}

\begin{itemize}
\item The \texttt{userInput} method is called by Main before any other method.
\end{itemize}

\newpage
\subsubsection* {Access Routine Semantics}

\noindent userInput():
\begin{itemize}
\item output: $\mathit{out} :=$ A dictionary (key:value pair) of parameters and their values. The keys are the parameters and the values are the user input.
\color{red}The following parameters are valid:

\begin{table}[h]
\color{red}
\begin{tabularx}{\textwidth}{llll}
\toprule {\bf Parameter} & {\bf Abbr.} & {\bf Description} & {\bf Example}\\
\midrule
\texttt{keyword} & \texttt{k} & The keyword to search Google Images with & \texttt{"software"}\\
\texttt{limit} & \texttt{l} & The maximum number of images to download & \texttt{10}\\
\texttt{safesearch} & \texttt{ss} & Whether or not to enable SafeSearch & N/A\\
\texttt{filetype} & \texttt{ft} & The file type of images to download & \texttt{"gif"}\\
\texttt{directory} & \texttt{d} & The directory to save images to & a path to a directory\\
\texttt{colour} & \texttt{c} & The colour of images to download & \texttt{"teal"}\\
\texttt{license} & \texttt{li} & The license of images to download & \texttt{"labeled-for-reuse"}\\
\texttt{imagetype} & \texttt{t} & The type of images to download &  \texttt{"clipart"}\\
\texttt{imageage} & \texttt{a} & The age of images to download &  \texttt{"past-year"}\\
\texttt{aspectratio} & \texttt{ar} & The aspect ratio of images to download &  \texttt{"square"}\\
\texttt{imagesize} & \texttt{is} & The size of images to download &  \texttt{">2MP"}\\
\texttt{serverhost} & \texttt{s} & The host of the server to download images to & \texttt{"moore.mcmaster.ca"}\\
\texttt{serverusername} & \texttt{u} & The username to log in to the given server & the user's MacID\\
\texttt{serverpassword} & \texttt{p} & The password to log in to the given server & the user's Moore password\\
\texttt{region} & \texttt{rg} & The geographical location of images to download & \texttt{"Canada"}\\
\texttt{whitelist} & \texttt{wl} & The website to download images from & \texttt{"mcmaster.ca"}\\
\texttt{blacklist} & \texttt{bl} & The website to exclude image download from & \texttt{"mcmaster.ca"}\\
\texttt{fromfile} & \texttt{ff} & The file to parse input arguments from & a path to a file\\
\bottomrule
\end{tabularx}
\caption{\bf Input Parameters}
\label{tab:inputParams}
\end{table}

\color{black}

\item exception: $\mathit{exc} :=$ (required parameter missing $\Rightarrow$ $\mathit{MissingParam} ~\vert$ improper format for a parameter $\Rightarrow$ $\mathit{InvalidParam}$)
\end{itemize}

\subsection* {Local Functions}

keywordFromFile: string $\rightarrow$ dict \\
keywordFromFile($\mathit{s}$) $\equiv$ Reads in file $\mathit{s}$ of parameters and their associated values\color{red}, populating fields with their default values if not found in the file\color{black}.\\
exception: $exc :=$ \st{(can't find file $\Rightarrow$ $\mathit{InvalidFileName} ~\vert$ can't read file $\Rightarrow$ $\mathit{UnableToParseFile} ~\vert$ required parameter missing $\Rightarrow$ $\mathit{MissingParam}~\vert$ improper format for a parameter $\Rightarrow$ $\mathit{InvalidParam}$)} \color{red}(error processing file $\Rightarrow \mathit{Exception} ~\vert$ \texttt{keyword} argument not found $\Rightarrow \mathit{ValueError}$)\color{black}

\newpage

\section*{Search Query Module}

\subsection* {Module}

SearchQuery

\subsection* {Uses}

None

\subsection* {Syntax}

\subsubsection* {Exported Types}

None

\subsubsection* {Exported Constants}

None

\subsubsection* {Exported Access Programs}

\begin{tabular}{| l | l | l | p{5cm} |}
\hline
\textbf{Routine name} & \textbf{In} & \textbf{Out} & \textbf{Exceptions}\\
\hline
buildURL & dict of string & string & None\\
\hline

\end{tabular}

\subsection* {Semantics}

\subsubsection* {Environmental Variables}

None

\subsubsection* {State Variables}

None

\subsubsection* {State Invariant}

None

\subsubsection* {Assumptions \& Design Decisions}

\begin{itemize}
\item All input arguments stored in the dictionary of strings are correct and in the proper format
\end{itemize}

\subsubsection* {Access Routine Semantics}

\noindent buildURL($dS$):
\begin{itemize}
\item output: $out :=$ \st{``https://www.google.com/search?q=\&espv=2\&sxsrf=ACYBGNSwqBUElVjmEWOTu3-\linebreak
mXPnReqFoLw:1581376760401\&source=lnms" + buildURLParam(dS) + ``isch\&sa=X\&ved=\linebreak
2ahUKEwiY7bzAj8jnAhUQjq0KHbXwBEYQ\_AUoAXoECBMQAw\&biw=838\&bih=880"} \color{red} A valid Google Images search URL with the \texttt{keyword}, \texttt{region}, \texttt{whitelist}, and the rest of the parameters formatted and encoded as necessary.\color{black}
\item exception: none
\end{itemize}

\newpage

\subsection* {Local Functions}

\noindent buildURLParam: dict of string $\rightarrow$ string \\
buildURLParam($\mathit{dS}$) $\equiv$ \st{$s$
such that:

$s =$ ``"

for all $i$ in $dS$:

$~~~~s = s + i$}

\color{red} A string consisting of the correct Google parameters for \texttt{aspectratio}, \texttt{colour}, \texttt{filetype}, \texttt{imageage}, \texttt{imagesize}, \texttt{imagetype}, and \texttt{license} as a comma-separated string.\color{black}

\newpage
\section*{Navigate Page Module}

\subsection* {Module}

NavigatePage

\subsection* {Uses}

None

\subsection* {Syntax}

\subsubsection* {Exported Types}

None

\subsubsection* {Exported Constants}

None

\subsubsection* {Exported Access Programs}

\begin{tabular}{| l | l | l | p{5cm} |}
\hline
\textbf{Routine name} & \textbf{In} & \textbf{Out} & \textbf{Exceptions}\\
\hline
getImageURL & string, $\mathbb{Z}$\color{red}, string& list of string & None \\
\hline

\end{tabular}

\subsection* {Semantics}

\subsubsection* {Environmental Variables}



\subsubsection* {State Variables}

None

\subsubsection* {State Invariant}

None

\subsubsection* {Assumptions \& Design Decisions}

\begin{itemize}
\item The link passed into the \texttt{getImageURL} method is correct and corresponds to a real link.
\item The user has setup the program correctly (correct chromedriver in right location, installed selenium, put correct location of Google Chrome binary).

\end{itemize}

\subsubsection* {Access Routine Semantics}

\noindent getImageURL($s$, $n$\color{red}, $b$\color{black}):
\begin{itemize}
\item output: $\mathit{out} :=$ a list ($urls$) of image URLs from $s$ where $\mathit{|urls| = n}$ \color{red}and none of the URLs are from the blacklisted site $b$\color{black}.
\item exception: \st{None} \color{red}($n \leq 0 \Rightarrow \mathit{ValueError} ~\vert$ OS not supported $\Rightarrow \mathit{Exception}$)\color{black}
\end{itemize}

\newpage

\section*{Output Format Module}

\subsection* {Module}

Output

\subsection* {Uses}

None

\subsection* {Syntax}

\subsubsection* {Exported Types}

None

\subsubsection* {Exported Constants}

None

\subsubsection* {Exported Access Programs}

\begin{tabular}{| l | l | l | p{5cm} |}
\hline
\textbf{Routine name} & \textbf{In} & \textbf{Out} & \textbf{Exceptions}\\
\hline
downloadImages & list of string, string, string & None & None\\
moveToServer & string, string, string, string, string & None & Exception\\
\hline

\end{tabular}

\subsection* {Semantics}

\subsubsection* {Environmental Variables}

\begin{itemize}
\item Screen
\end{itemize}

\subsubsection* {State Variables}

None

\subsubsection* {State Invariant}

None

\subsubsection* {Assumptions \& Design Decisions}

\begin{itemize}
\item \st{The first string parameter (the download location) in the \texttt{downloadImages} method is optional and has a default value (``./Images").}
\item The \texttt{createDirectory} method creates a directory 
\color{red}called $\mathit{key}$ in the directory $\mathit{loc}$ and raises an $\mathit{OSError}$ if the directory can't be created. \color{black} \st{with the file path $\mathit{loc} + ``/" + \mathit{key}$.}
\end{itemize}

\subsubsection* {Access Routine Semantics}

\noindent downloadImages($lst$, $key$, $loc$):
\begin{itemize}
\item output: save each image represented by a string in $lst$ and save it \color{red} with the name $key$ followed by an appropriate number \color{black} to the folder created by \color{red}calling \color{black} the \texttt{createDirectory} method.
\item exception: none
\end{itemize}

\noindent \textcolor{red}{moveToServer($key$, $direc$, $shost$, $suser$, $spass$):}
\begin{itemize}
\item \textcolor{red}{output:} \textcolor{red}{Puts the directory at src/Images/$key$ on $suser$@$shost$:$key$/}
\item \textcolor{red}{exception:} \textcolor{red}{If problem with putting directory on server $\rightarrow$ Exception}
\end{itemize}

\subsection* {Local Functions}

\noindent \textcolor{red}{createSSH($key$, $direc$, $shost$, $suser$, $spass$): 
string x string x string x string x string $\rightarrow$ SSHClient} \\
\textcolor{red}{createSSH($key$, $direc$, $shost$, $suser$, $spass$): 
An SSHClient object that can be used to transport files to and from a server} \\

\noindent \textcolor{red}{deleteLocalImageFolder($direc$, $key$): 
string x string  $\rightarrow$ $\epsilon$} \\
\textcolor{red}{createSSH($direc$, $key$): 
Deletes local copy of images downloaded from src/Images/$key$} \\

\noindent createDirectory(\st{$loc$, $key$} \color{red}$dir$\color{black}): $\epsilon$ \\
createDirectory((\st{$loc$, $key$} \color{red}$dir$\color{black}) $\equiv$ Null

\medskip

\color{red}
\noindent getRequest($img$): byte string \\
getRequest($img$) $\equiv$ The data of the image denoted by $img$.
\color{black}

\newpage

\section*{Main Module}

\subsection* {Module}

Main

\subsection* {Uses}

Input, SearchQuery, NavigatePage, Output

\subsection* {Syntax}

\subsubsection* {Exported Types}

None

\subsubsection* {Exported Constants}

None

\subsubsection* {Exported Access Programs}

None

\subsection* {Semantics}

\subsubsection* {Environmental Variables}

None

\subsubsection* {State Variables}

None

\subsubsection* {State Invariant}

None

\subsubsection* {Assumptions \& Design Decisions}

\begin{itemize}
    \item The main function in the Main module is used to call all the other modules in order to run them in the right order. The main function will call the modules in the following order: Input, SearchQuery, NavigatePage, Output.
\end{itemize}

\subsubsection* {Access Routine Semantics}

None

\subsection* {Local Functions}

\noindent main: $\epsilon \rightarrow \epsilon$  \\
main() $\equiv$ Null\\

\end{document}
