\documentclass{article}

\usepackage{booktabs}
\usepackage{tabularx}
\usepackage[backend=bibtex]{biblatex}
\usepackage{hyperref}
\usepackage{soul,xcolor}

\bibliography{bibfile}

\hypersetup{
    colorlinks=true,
    linkcolor=blue,
    filecolor=magenta,      
    urlcolor=cyan,
}

\title{SE 3XA3: Development Plan\\Google Images Downloader}

\author{Team 201, CAS Dream Team
		\\ Sam Crawford, crawfs1, 400129435
		\\ Joshua Guinness, guinnesj, 400134735
		\\ Nicholas Mari, marin, 400132494
}

\date{January 31, 2020}

% \input{../Comments}

\begin{document}

\setstcolor{black}

\maketitle

\tableofcontents

\newpage

\section{Introduction}
This document outlines many aspects of how we intend to develop our product, 
including our team roles, technological details, and project management.

\section{Team Meeting Plan}
Meetings will take place every lab session, which is twice a week on Tuesday and 
Wednesday from 2:30 - 4:30 pm in ITB 236, when possible. \st{If extra meetings are 
needed,} \color{red} Extra meetings will be scheduled when needed on a case-by-case 
basis at \color{black} a time and location that fits for everyone in the group.

The roles individuals take in meetings are outlined below in Section~\ref{roles}. 

An agenda will be set by the project manager (Joshua Guinness) before the 
meeting and will be sent to the Facebook group chat so everyone in the group is 
aware. The project manager will chair the meeting and will ensure that this 
agenda is adhered to during the meetings. Any next steps, deliverables, or actions 
that need to take place will be agreed to by the individuals at the end of the meeting. 

\section{Team Communication Plan}
In order to properly communicate between team members on important issues 
related to code and development, we will make use of \st{git issue} \color{red} the 
git issue tracker when necessary\color{black}. This also provides us 
with a logged record of all the issues and problems that occurred during the development process. \color{red}Other communication, like discussing tasks or diagnosing bugs, will also be done via Discord and Facebook Messenger.\color{black}

For team meetings and general planning, our team will make use of Discord for 
remote meetings and Facebook Messenger for meeting planning and coordination.

\section{Team Member Roles}
\label{roles}
All members will also act as developers, designing and implementing the code for 
our program.

\subsection{Joshua Guinness - Project Manager}
\begin{itemize}
\item Responsible for meeting minutes and other supporting documentation
\item Will ensure project deadlines are adhered to and everyone is kept on track
\end{itemize}

\subsection{Nicholas Mari - Software Tester}
\begin{itemize}
\item Responsible for testing the software and ensuring it meets our requirements
\end{itemize}

\subsection{Sam Crawford - Git Specialist}
\begin{itemize}
\item Responsible for fixing issues arising with Git, such as merge conflicts 
\end{itemize}

\section{Git Workflow Plan}
We will be using trunk-based development for our workflow. We will branch off 
of the master branch only when needed for implementing a specific feature; 
otherwise we will be working in the master branch, especially for 
smaller updates like modifying documentation. This will allow us to avoid 
frequent and messy merge conflicts. As we are just getting started, and since 
we're working with a small team, using this method of code development will 
allow us to develop our code and write our documentation quickly with minimal 
wait times or delays \cite{trunkbased}.

\section{Proof of Concept Demonstration Plan}
In order to provide a proof of concept for our product, the basic functionality
of the application will be demonstrated. A Python script will be created that can
\color{red}d\color{black}ownload multiple images from the Internet given a keyword. 
However, as the proof of concept intends to focus on the basic functions, the 
additional features such as input flags \color{red}and \color{black}white-listing 
\st{, a vetting process, or a GUI} will not be implemented in this version. 
\color{red} At the onset of this project, Google had changed their method of 
handling images; instead of having the image URLs directly in the HTML, they 
resolved it dynamically using JavaScript. This means that the original 
implementation no longer works. Since mitigating this issue will be a significant 
hurdle, the proof of concept will only focus on creating a product that allows the 
user to download images. This proof of concept will demonstrate that a solution to 
this problem is still feasible, despite the change in Google's service.
\color{black}

\section{Technology}
Programming Language: Python 3 \\
IDEs: Visual Studio/Sublime Text \\
Testing Framework: pytest \\
Document Generation: doxygen \\
Linters: \st{pylint/}flake8

\section{Coding Style}
We will be basing our style guide off of Google’s style guide for Python 
\color{red} \cite{pythonguide}\color{black}, with some modifications. Linting will 
be done \st{as per our own discretion, either with pylint or flake8; this is 
subject to revision in the future and will possibly be standardized.} \color{red} 
with flake8, instead of the pylint used by Google. \color{black}We will not be 
considering Python 2 compatibility, type annotating 
functions or variables, or using shebang lines, unless we decide it is 
advantageous during the development process. All functions, variables, classes, 
modules, etc. will be named with camel case convention, with the first letter’s 
capitalization depending on the type (ie. classes will use 
\texttt{UpperCamelCase} and functions will use \texttt{lowerCamelCase}); the 
rationale for this is improving readability and reducing long function names.

\section{Project Schedule}
The \href{https://gitlab.cas.mcmaster.ca/guinnesj/google-images-downloader/blob/master/ProjectSchedule/Gantt-Chart.pdf}{Gantt Chart} can be found in the project schedule folder on the repository or by clicking the link in this section.

\section{Project Review}

\printbibliography{}

\begin{table}[h]
\caption{Revision History} \label{TblRevisionHistory}
\begin{tabularx}{\textwidth}{lll}
\toprule
\textbf{Date} & \textbf{Developer(s)} & \textbf{Change}\\
\midrule
Jan 28 & Joshua & Added team member information\\
Jan 29 & Sam & Added bibfile\\
Jan 29 & Sam & Added section content\\
Jan 29 & Sam & Added table of contents and Introduction\\
Jan 29 & Nick & Added Gantt Chart link\\
Jan 30 & Nick & Updated Gantt Chart link\\
Jan 30 & Joshua & Spelling, grammar, sentence structure, formatting \\
\color{red}Apr 5&\color{red}Sam&\color{red}Revisions for Rev 0\\
\bottomrule
\end{tabularx}
\end{table}

\end{document}